\documentclass[11pt]{article}

\usepackage{amsfonts}
\usepackage{amscd}
\usepackage{amsmath}
\usepackage{amsthm}
\usepackage{amssymb}
\usepackage{setspace}
\usepackage{epsfig}
\usepackage{fullpage}
\usepackage{comment}
\usepackage{mathtools}
\usepackage{eucal}

\newcommand{\vs}{\vspace{0.2in}}
\newcommand{\vvs}{\vspace{0.1in}}

\topmargin -.5in
\textheight 9in
\oddsidemargin -.25in
\evensidemargin -.25in
\textwidth 7in

\begin{document}
\tt
\centerline{MATH 150: Quiz 6}
\vs

Name: \underline{\hspace{5cm}}

\medskip

\begin{enumerate}
% -------------------------------------------------------------
% -------------------------------------------------------------
% -------------------------------------------------------------
\item
Graph the line represented by the equation. 
(1 point)
\begin{align*}    
y = -5x - 3
\end{align*}
\vspace{3in}

\item
Use slope-intercept form to write the equation of a line passing through the given 

point and having the given slope. 
(1 point)
\begin{align*}
    P(-3, 1)\text{;}\hspace{0.5cm} m = 1
\end{align*}

% -------------------------------------------------------------
% -------------------------------------------------------------
% -------------------------------------------------------------
\newpage

\item
Find the slope and the y-intercept of the line determined by the given equation. 

(2 points)
\begin{align*}
-6x + 12y = 36
\end{align*}

\vspace{3in}

\item
Consider the following pair of equations. 
(2 points)
\begin{align*}
    x = 6y + 8
    \text{,} \hspace{1cm}
    y = -6x + 7
\end{align*}
\vs

\begin{enumerate}
    \item 
    The slope of the line $x = 6y + 8$ is
    \underline{\hspace{3cm}}
    \vs
    
    \item 
    The slope of the line $y = -6x + 7$ is
    \underline{\hspace{3cm}}
    \vs

    \item Determine whether the graphs of the pair of equations 
    
    are parallel, perpendicular, or neither. 

% -------------------------------------------------------------
% -------------------------------------------------------------
% -------------------------------------------------------------
\newpage

\item 
Convert the general form of the circle given into standard form. 
(2 Points)
\begin{align*}
    P(5, 0)
    \text{,}
    \hspace{0.5cm}
    Q(6, -6)
\end{align*}

\vspace{3in}

\item 
Convert the general form of the circle given into standard form. 
(2 Points)
\begin{align*}
    x^2 + y^2 + 2x + 18y + 57 = 0
\end{align*}

    
\end{enumerate}




\end{enumerate}

\end{document}

